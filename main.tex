% !TEX TS-program xelatex
% !TEX encoding = UTF-8 Unicode
% -*- coding: UTF-8; -*-
% vim: set fenc=utf-8

%%%%%%%%%%%%%%%%%%%%%%%%%%%%%%%%%%%%%%%%%%%%%%
%%  FileName: main.tex
%%  Author:   Timozer
%%  E-mail:   zhenyuwang94@gmail.com
%%%%%%%%%%%%%%%%%%%%%%%%%%%%%%%%%%%%%%%%%%%%%%

\documentclass[no-math]{beamer}

\usepackage[UTF8, noindent]{ctexcap}

\usetheme{TIMOZER}

\title{三峡中心403例会幻灯片\TeX{}模板}
\subtitle{\LaTeXe{} Beamer 模板 \textsc{Version 0.1}}
\date{\today}
\author{王震宇\quad \texttt{zhenyuwang94@gmail.com}}
\institute{计算机学院}

\begin{document}
    \begin{frame}
        \titlepage
    \end{frame}
    \begin{frame}{目录}
        \tableofcontents
    \end{frame}
    
    \section{介绍}
    \begin{frame}{介绍}
        该 \LaTeXe{} \textsc{Beamer} 模板是我基于吴比师兄开例会时使用的 PPT 模板
        制作而成. 基本布局和他的模板类似, 有一些其他的是我自己添加的, 比如说每
        页下面的导航条, 日期和页码.

    \end{frame}
    \section{使用方法}
        \begin{frame}{使用方法}
            在开始创建你的 Beamer 文档之前, 请将该模板所有的文件放到你的文件夹的根目录
            里.
            \begin{block}{模版文件}
                \begin{itemize}
                    \item beamerthemeTIMOZER.sty
                    \item beamercolorthemeTIMOZER.sty
                    \item beamerfontthemeTIMOZER.sty
                    \item beamerinnerthemeTIMOZER.sty
                    \item beamerouterthemeTIMOZER.sty
                \end{itemize}
            \end{block}
            此外, 还有 Makefile, README.md, main.tex \ldots 等文件不是该模板的主要文件,
            可以删除.
        \end{frame}

        \begin{frame}[fragile]{使用方法}
            新建 Beamer 文档, 这里假设文档名字为 \texttt{demo.tex}.\\
            接下来引入该模板, 如下:
            \begin{minted}{tex}
                \documentclass[no-math]{beamer}
                \usepackage[UTF8, noindent]{ctexcap}
                \usetheme{TIMOZER}
                \title{三峡中心403例会幻灯片\TeX{}模板}
                \subtitle{\LaTeXe{} Beamer 模板 \textsc{Version 0.1}}
                \date{\today}
                \author{王震宇\quad \texttt{zhenyuwang94@gmail.com}}
                \institute{计算机学院}
            \end{minted}
        \end{frame}
        \begin{frame}{使用方法}
            正如你在代码中看到的, 除了使用文档类 \texttt{beamer}, 为了使用中文, 还得加载
            \texttt{ctexcap} 宏包, 这个宏包为我们提供了中文支持, 并且将一些环境的名字也换成了
            中文的.

            接下来就是使用 \textsc{Timozer} 主题了.

            最后输入你的题目, 如果有小标题, 也可以输入. 你的名字, 学院, 日期等等.
            然后将你的主要内容放入 \texttt{document} 环境中.
        
        \end{frame}

    \section{版本}
        \begin{frame}{\textsc{Version 0.1}}
            \begin{block}{Version 0.1}
                2018年 2月 5日 星期一 12时32分51秒 CST
            \end{block}
        \end{frame}

    \section{致谢}
        \begin{frame}{致谢}
            首先感谢吴比师兄为我提供 PPT 模板.
        \end{frame}
    \begin{frame}{大标题}
        只有一个标题
        内容是垂直居中的 \cite{Teney_2017_CVPR}
    \end{frame}

    \section{参考文献}
    \begin{frame}{参考文献}
        \bibliographystyle{apalike}
        \bibliography{403beamer}
    \end{frame}
\end{document}
